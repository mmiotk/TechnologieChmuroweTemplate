\documentclass[12pt,a4paper]{article}
\usepackage[utf8]{inputenc}
\usepackage{dsfont} 
\usepackage[polish]{babel}
\usepackage{amsmath}
\usepackage{graphicx}
\usepackage[top=1in, bottom=1.5in, left=1.25in, right=1.25in]{geometry}

\usepackage{subfig}
\usepackage{multirow}
\usepackage{multicol}
\graphicspath{{Imagens/}}
\usepackage{xcolor,colortbl}
\usepackage{float}

\newcommand \comment[1]{\textbf{\textcolor{red}{#1}}}

%\usepackage{float}
\usepackage{fancyhdr} % Required for custom headers
\usepackage{lastpage} % Required to determine the last page for the footer
\usepackage{extramarks} % Required for headers and footers
\usepackage{indentfirst}
\usepackage{placeins}
\usepackage{scalefnt}
\usepackage{xcolor,listings}
\usepackage{textcomp}
\usepackage{color}
\usepackage{verbatim}
\usepackage{framed}

\definecolor{codegreen}{rgb}{0,0.6,0}
\definecolor{codegray}{rgb}{0.5,0.5,0.5}
\definecolor{codepurple}{HTML}{C42043}
\definecolor{backcolour}{HTML}{F2F2F2}
\definecolor{bookColor}{cmyk}{0,0,0,0.90}  
\color{bookColor}

\lstset{upquote=true}

\lstdefinestyle{mystyle}{
	backgroundcolor=\color{backcolour},   
	commentstyle=\color{codegreen},
	keywordstyle=\color{codepurple},
	numberstyle=\numberstyle,
	stringstyle=\color{codepurple},
	basicstyle=\footnotesize\ttfamily,
	breakatwhitespace=false,
	breaklines=true,
	captionpos=b,
	keepspaces=true,
	numbers=left,
	numbersep=10pt,
	showspaces=false,
	showstringspaces=false,
	showtabs=false,
}
\lstset{style=mystyle}

\newcommand\numberstyle[1]{%
	\footnotesize
	\color{codegray}%
	\ttfamily
	\ifnum#1<10 0\fi#1 |%
}

\definecolor{shadecolor}{HTML}{F2F2F2}

\newenvironment{sqltable}%
{\snugshade\verbatim}%
{\endverbatim\endsnugshade}

% Margins
\addtolength{\footskip}{0cm}
\addtolength{\textwidth}{1.4cm}
\addtolength{\oddsidemargin}{-.7cm}

\addtolength{\textheight}{1.6cm}
%\addtolength{\topmargin}{-2cm}

% paragrafo
\addtolength{\parskip}{.2cm}

% Set up the header and footer
\pagestyle{fancy}
\rhead{\hmwkAuthorName} % Top left header
\lhead{\hmwkClass: \hmwkTitle} % Top center header
\rhead{\firstxmark} % Top right header
\lfoot{Jan Kowalski} % Bottom left footer
\cfoot{} % Bottom center footer
\rfoot{} % Bottom right footer
\renewcommand{\headrulewidth}{1pt}
\renewcommand{\footrulewidth}{1pt}

    
\newcommand{\hmwkTitle}{Tytuł projektu z technologii chmurowych} % Tytuł projektu
\newcommand{\hmwkDueDate}{\today} % Data 
\newcommand{\hmwkClass}{Technologie chmurowe} % Nazwa przedmiotu
\newcommand{\hmwkAuthorName}{Jan Kowalski} % Imię i nazwisko

% trabalho 
\begin{document}
% capa
\begin{titlepage}
    \vfill
	\begin{center}
	\hspace*{-1cm}
	\vspace*{0.5cm}
    \includegraphics[scale=0.55]{imagens/loga.png}\\
	\textbf{Uniwersytet Gdański \\ [0.05cm]Wydział Matematyki, Fizyki i Informatyki \\ [0.05cm] Instytut Informatyki}

	\vspace{0.6cm}
	\vspace{4cm}
	{\huge \textbf{\hmwkTitle}}\vspace{8mm}
	
	{\large \textbf{\hmwkAuthorName}}\\[3cm]
	
		\hspace{.45\textwidth} %posiciona a minipage
	   \begin{minipage}{.5\textwidth}
	   Projekt z przedmiotu technologie chmurowe na kierunku informatyka profil praktyczny na Uniwersytecie Gdańskim.\\[0.1cm]
	  \end{minipage}
	  \vfill
	%\vspace{2cm}
	
	\textbf{Gdańsk}
	
	\textbf{\hmwkDueDate}
	\end{center}
	
\end{titlepage}

\newpage
\setcounter{secnumdepth}{5}
\tableofcontents
\newpage

\section{Opis projektu}
\label{sec:Project}

Tutaj wypisujemy szczegółowe informacje o celu powstania projektu (jakaś historyjka, że pewna firma/osoba potrzebuje bazy, która ...)

\subsection{Opis architektury - 8 pkt}
\label{sec:introduction}
Powinna zawierać opis architektury aplikacji opartej na Kubernetes, w tym opis klastrowego systemu zarządzania kontenerami, wykorzystanych modułów i komponentów


\subsection{Opis infrastruktury - 6 pkt}
\label{sec:Users}

Powinien zawierać informacje na temat środowiska, w którym aplikacja będzie działać, a także wykorzystanych narzędzi i platformy chmurowe. Ważne jest również zwrócenie uwagi na wykorzystanie zasobów, takich jak sieci i pamięci masowej.

\subsection{Opis komponentów aplikacji - 8 pkt}
\label{sec:FunctionalConditions}

Powinna zawierać informacje na temat komponentów aplikacji, takich jak serwisy, aplikacje i bazy danych. W szczególności należy zwrócić uwagę na sposoby ich wdrażania, konfiguracji i zarządzania.

\subsection{Konfiguracja i zarządzanie - 4 pkt}
\label{sec:NonFunctionalConditions}

Powinna zawierać informacje na temat konfiguracji i zarządzania aplikacją na poziomie klastra Kubernetes.

\subsection{Zarządzanie błędami - 2 pkt}
\label{sec:ERD} 

Powinna zawierać informacje na temat sposobów zarządzania błędami aplikacji oraz sposobów monitorowania i reagowania na awarie.


\subsection{Skalowalność - 4 pkt}
\label{sec:ExamplesSection}

Skalowalność jest kluczowa w architekturze aplikacji opartej na Kubernetes. Należy opisać, jak aplikacja może być skalowana, w jaki sposób skalowanie jest monitorowane i jakie narzędzia są wykorzystywane w tym celu.

\subsection{Wymagania dotyczące zasobów - 2 pkt}
\label{sec:ExampleTables}

Powinna zawierać informacje na temat wymagań dotyczących zasobów dla każdego komponentu aplikacji, takie jak ilość pamięci RAM, CPU, miejsce na dysku, itp. Należy również opisać, jakie są oczekiwania dotyczące wydajności i czasu odpowiedzi dla aplikacji.


\subsection{Architektura sieciowa - 4 pkt}
\label{sec:ExampleResults}

powinna zawierać informacje na temat architektury sieciowej aplikacji, w tym sposobu konfiguracji sieci w klastrze Kubernetes, wykorzystywanych protokołów i narzędzi do zarządzania siecią.

\noindent
\textbf{Każdy odnośnik, który znajdzie się w literaturze musi mieć swoje odwołanie w projekcie. - 2 pkt} 

\bibliographystyle{amsplain}
\bibliography{references.bib}
\nocite{*}

\end{document}